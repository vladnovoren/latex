\documentclass[a4paper, 14pt]{article}
\usepackage[utf8]{inputenc}
\usepackage[russian]{babel}
\usepackage{amsfonts}
\usepackage{mathtools}
\usepackage{indentfirst}
\usepackage{amsmath}
\usepackage{amssymb}
\usepackage[left=15mm, top=10mm, right=15mm, bottom=20mm]{geometry}

\date{}
\begin{document}
\title{Лекция 10. Степенные ряды (продолжение).}
\maketitle

$$
\sum\limits_{n=0}^{\infty}c_n(z-z_0)^n \eqno (1)
$$

\textbf{Теорема 3 (Абель).} Если степенной ряд (1) сходится в точке $z_1 \neq z_0$, то он сходится равномерно на отрезке $\Delta_{z_0, z_1}=\left\{z \in \mathbb{C}:z = z_0 + t(z_1 - z_0),\:t \in [0, 1]\right\}$.

$\blacktriangle$ Последовательность $\{t^n\}$ монотонна $\forall t \in [0, 1]$ и равномерно ограничена на $[0, 1]$. По условию\\ $\sum\limits_{n=0}^{\infty}c_n (z_1-z_0)^n$ сходится. Поэтому по прищнаку Абеля для функциональных рядов ряд вида $\sum\limits_{n=0}^{\infty}c_n (z_1-z_0)^n t^n$ сходится равномерно на $[0,1]$. Сделав замену $t=\frac{z-z_0}{z_1-z_0}$, получим ряд (1). Следовательно, ряд (1) сходится равномерно на $\Delta_{z_0, z_1}$.\:$\blacksquare$

\textbf{Замечание.} Если $z \in B_R(z_0)$, то теорема Абеля следует из теоремы 2.

\textbf{Лемма.} Степенные ряды
$$
\sum\limits_{n=0}^{\infty}c_n(z-z_0)^n, \quad \sum\limits_{n=1}^{\infty}n c_n(z-z_0)^{n-1}, \quad \sum\limits_{n=0}^{\infty}\frac{c_n}{n+1}(z-z_0)^{n+1}
$$
имеют одинаковый радиус сходимости.

$\blacktriangle$ Обозначим радиус сходимости первого ряда через $R$. Т.к. $\lim\limits_{n \to \infty}\sqrt[n]{n}=1$, то множества частичных пределов последовательностей $\left\{\sqrt[n]{n|c_n|}\right\}$ и $\left\{\sqrt[n]{|c_n|}\right\}$ совпадают и, значит,
$$
\varlimsup_{n \to \infty}\sqrt[n]{n|c_n|}=\varlimsup_{n \to \infty}\sqrt[n]{|c_n|}
$$
По формуле Коши-Адамара радиус сходимости ряда $\sum\limits_{n=0}^{\infty}n c_n (z-z_0)^n$ равен $R$, следовательно и радиус сходимости второго ряда равен $R$. Первый ряд получается из третьего почленным дифференцированием $\Rightarrow$ радиусы сходимости этих рядов совпадают.\:$\blacksquare$

\textbf{Теорема 4.} Если $f(z)=\sum\limits_{n=0}^{\infty}c_n (z-z_0)^{n}$ - сумма степенного ряда с радиусом сходимости $R>0$, то $f$ бесконечно дифференцируема в круге сходимости $B_R(z_0)$, причем в $\forall m \in \mathbb{N}$ в этом круге имеет место равенство
$$
f^{(m)}(z)=\sum\limits_{n=m}^{\infty}n(n-1)\ldots(n-m+1)c_n (z-z_0)^{n-m}
$$

$\blacktriangle$ По лемме 1 при дифференцировании радиус сходимости степенного ряда не меняется, поэтому достаточно доказать утверждение для $m=1$ и применить индукцию. Без ограничения общности можно считать $z_0=0$.

Возьмем $w \in B_R(0)$ и покажем, что производная функции $f(z)=\sum\limits_{n=0}^{\infty}c_n z^n$ в точке {w} равна числу $l$, где
$$
l=\sum\limits_{n=0}^{\infty}n c_n w^{n-1}
$$
Зафиксируем $r$ так, чтобы $|w|<r<R$. Для $z \neq w$, $|z|<r$ рассмотрим разность
$$
\frac{f(z)-f(w)}{z-w}-l=\sum\limits_{n=0}^{\infty}c_n \left(\frac{z^n-w^n}{z-w} - n w^{n-1}\right)=
\sum\limits_{n=0}^{\infty}c_n\left(z^{n-1}+z^{n-2}w+\ldots+z w^{n-2}+w^{n-1}-n w^{n-1}\right) \eqno (2)
$$
Перепишем выражение в скобках в следующем виде:
\begin{multline*}
(z^{n-1}-w^{n-1})+w(z^{n-2}-w^{n-2})+\ldots+w^{n-2} (z-w)= \\= (z-w)\left[(z^{n-2}+z^{n-3}w+\ldots+w^{n-2})+w(z^{n-3}+z^{n-4}w+\ldots+w^{n-3})+\ldots+w^{n-2}\right]
\end{multline*}
Т.к. $(n-1)+(n-2)+\ldots+1=\frac{n(n-1)}{2}$, то для $n$-го члена (2) справедливо
$$
\left|c_n (z^{n-1}+z^{n-2}w+\ldots+w^{n-1}-n w^{n-1})\right| \leqslant|z-w||c_n|\frac{n(n-1)}{2}r^{n-2}
$$
Ряд $\sum\limits_{n=2}^{\infty}|c_n|\frac{n(n-1)}{2}r^{n-2}$ сходится, т.к. $r<R$ и дважды дифференцированный ряд имеет радиус сходимости $R$. Следовательно
$$
\left|\frac{f(z)-f(w)}{z-w}-l\right| \leqslant |z-w| \sum\limits_{n=2}^{\infty}\frac{n(n-1)}{2}|c_n|r^{n-2} \to 0,\;z \to w
$$
и, значит, $\exists \lim\limits_{z \to w}\frac{f(z)-f(w)}{z-w}=l$.\:$\blacksquare$

\textbf{Следствие 1.} Степенной ряд (1) с радиусом сходимости $R>0$ имеет в круге сходимости первообразную
$$
F(z)=C+\sum\limits_{n=0}^{\infty}\frac{c_n}{n+1}(z-z_0)^{n+1}
$$

\textbf{Следствие 2.} Если степенной (1) имеет радиус сходимости $R>0$, то его коэффициенты однозначно определяются по формуле
$$
c_m=\frac{f^{(m)}(z_0)}{m!},\:m=0,1,2,\ldots
$$

\textbf{Следствие 3 (теорема единственности).} Если степенные ряды $\sum\limits_{n=0}^{\infty}a_n(z-z_0)^n \text{ и }\sum\limits_{n=0}^{\infty}b_n(z-z_0)^n$ сходятся в некотором круге $B_{\delta}(z_0)$, и их суммы в $B_{\delta}(z_0)$ совпадают, то $a_n=b_n,\:n=0,1,2,\ldots\;$.

\textbf{Определение.} Если функция $f$ в точке $z_0$ имеет производные всех порядков, то степенной ряд вида
$$
\sum\limits_{n=0}^{\infty}\frac{f^{(n)}(z_0)}{n!}(z-z_0)^n
$$
называется рядом Тейлора функции $f$ в точке $z_0$. Для $z_0=0$ называется рядом Маклорена.

\textbf{Действительные степенные ряды. Представление функций рядом Тейлора.}

Будем говорить, что функция $f$ на промежутке $I \subset \mathbb{R}$ представима в виде ряда $\sum\limits_{n=0}^{\infty}a_n(x-x_0)^n$, если ряд сходится на $\I$ и $f$ является его суммой на $I$. Промежуток $(x_0-R,x_0+R)$ назовём интервалом сходимости степенного ряда.
\end{document}
